\chapter{Introduction}\label{CH:introduction}


\section{Section}\label{Sect:test}
Welcome to the Montana State University electronic Thesis/Dissertation (ETD) \LaTeX{} template.  In this chapter various sections, subsections, and subsubsections are created and filled with random text).  In Ch.~\ref{CH:theory} methods to write equations and how to include figures and tables are explored.  Finally, conclusions are drawn in Ch.~\ref{conclusion}.

\subsection{Subsection}\label{Sect:testsub}
\lipsum[2] % Random text

\subsubsection{Subsubsection}\label{Sect:testsubsub}
\lipsum[3] % Random text

\longsubsection{Subsection with a very very very very very very very}{very very very very very very very very very long title}\label{Sect:longsub}
For long subsection titles use the command \verb|\longsubsection{#1}{#1}|, where \#1 is the first line of the long title, and \#2 is the second line of the long title. You can also pass an optional argument to this command that puts a shorter title in the table of contents as shown by the subsection below.

\longsubsection[Another subsection with a very long title]{Another subsection with a very very very very very very}{very very very very very very very very very long title}\label{Sect:longsub}
The are \textbf{not} similar commands for sections and subsubsections as these are not specified in the MSU style guide.  


